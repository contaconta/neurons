%% -*- mode: latex; mode: reftex; mode: flyspell; TeX-master: "top.tex"; -*-

%gg 20121023 - strengthened the findings
We present a fully automatic method  to track and quantify the morphodynamics of
differentiating neurons  in fluorescence  time-lapse datasets.  Our  approach is
capable of  robustly detecting, tracking,  and segmenting all the  components of
the neuron including the nucleus, soma, neurites, and filopodia.  It is designed
to be extremely efficient, capable of processing each image in approximately two
seconds  on a  conventional notebook  computer.   To validate  our approach,  we
analyzed 510 neuronal differentiation videos comprising 7,298 neurons in which a
set of  genes was  perturbed using RNA  interference.  Our  completely automated
analysis  quantifies  and  confirms   morphodynamic  behaviors  which  had  been
previously observed by biologists but  never measured.  Finally, we are first to
analyze temporally the  behavior of neurons, leading to  findings not measurable
through  state-of-the-art static  analysis. All  our findings  are statistically
significant, with a p-value $\ll 0.05$.

%% We  present  a  fully  automatic  method to  track  and  quantify  the
%% morphodynamics of  differentiating neurons in  fluorescence time-lapse
%% datasets.     Our  approach is capable of robustly detecting,
%% tracking, and segmenting all the  components of the neuron including 
%% the nucleus, soma, neurites, and filopodia.  It is designed to be extremely efficient, capable of processing
%% each image in approximately two seconds on a conventional notebook
%% computer.  To validate our approach, we  analyzed  neuronal
%% differentiation videos in  which a set of genes  was perturbed using
%% RNA interference. Our analysis quantifes and confirms morphodynamic behaviors 
%% which had been previously observed by biologists but never measured.
%% Finally, we present new observations on the behavior of neurons made
%% possible by our analysis which could not be discovered through static analyis.


\begin{keywords}
Molecular and cellular screening; Image sequence processing; Fluorescence microscopy
\end{keywords}

%  whose
%measurements were not feasible on a large scale.  Finally, using our
%quantitative analysis we make

%our dynamic
%quantitative analysis allows us to  make new observations that are not
%visible to a human observer.


%% we  use it to  perform an siRNA  screen and
%% successfully reproduce the findings of~\cite{}, which was performed at
%% steady-state under less stressful conditions [OLVIER, HELP US SAY THIS
%%   MORE  ELOQUENTLY].   Finally, we  apply  our  approach  to make  new
%% observations   based  on  dynamic   quantitative  analysis   that  was
%% previously impossible.

%% While  previous efforts
%% have tracked the soma and nucleus,  our approach is the first to fully automatically track
%% and reconstruct the neurites as  they expand, branch and collapse, and
%% to dynamically detect fine filopodia structures.

%We present a method to analyze and extract statistics from large-scale
%video sequences  of in-vitro neurons.   Our system is able  to detect,
%track, segment  and reconstruct each individual neuron  present on the
%videos.   We show how  such system  can be  used to  find correlations
%between neuron behaviour and the proteins used to mutate them.

%high throughput automation

%confirms pre-existing findings

%yields new understandings about the dynamic morphologies

%efficient, tracks, segments, and tracks neurites

