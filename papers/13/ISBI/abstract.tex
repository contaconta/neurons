%% -*- mode: latex; mode: reftex; mode: flyspell; TeX-master: "top.tex"; -*-

We  present  a  fully  automatic  method to  track  and  quantify  the
morphodynamics of  differentiating neurons in  fluorescence time-lapse
microscopy  datasets.    While  previous  efforts   have  successfully
quantified the dynamics of organelles  such as the cell body, nucleus,
or chromosomes of  cultured cells, neurons have proved  to be uniquely
challenging  due to  their  highly deformable  neurites which  expand,
branch, and collapse.  Our  approach is capable of robustly detecting,
tracking, and segmenting all the  components of each neuron present in
the sequence including the nucleus, soma, neurites, and filopodia.  To
meet  the   demands  required  for   high-throughput  processing,  our
framework is designed to be extremely efficient, capable of processing
a single image in approximately two seconds on a conventional notebook
computer.   For  validation  of  our approach,  we  analyzed  neuronal
differentiation datasets in  which a set of genes  was perturbed using
RNA interference. Our analysis confirms previous quantitative findings
measured  from   static  images,  as  well   as  previous  qualitative
observations of  morphodynamic phenotypes that could  not be measured
on a large scale. Finally,  we present new observations about the 
behavior of  neurons made  possible by our quantitative  analysis, which
are not immediately obvious to a human observer.

%  whose
%measurements were not feasible on a large scale.  Finally, using our
%quantitative analysis we make

%our dynamic
%quantitative analysis allows us to  make new observations that are not
%visible to a human observer.


%% we  use it to  perform an siRNA  screen and
%% successfully reproduce the findings of~\cite{}, which was performed at
%% steady-state under less stressful conditions [OLVIER, HELP US SAY THIS
%%   MORE  ELOQUENTLY].   Finally, we  apply  our  approach  to make  new
%% observations   based  on  dynamic   quantitative  analysis   that  was
%% previously impossible.

%% While  previous efforts
%% have tracked the soma and nucleus,  our approach is the first to fully automatically track
%% and reconstruct the neurites as  they expand, branch and collapse, and
%% to dynamically detect fine filopodia structures.

%We present a method to analyze and extract statistics from large-scale
%video sequences  of in-vitro neurons.   Our system is able  to detect,
%track, segment  and reconstruct each individual neuron  present on the
%videos.   We show how  such system  can be  used to  find correlations
%between neuron behaviour and the proteins used to mutate them.

%high throughput automation

%confirms pre-existing findings

%yields new understandings about the dynamic morphologies

%efficient, tracks, segments, and tracks neurites

