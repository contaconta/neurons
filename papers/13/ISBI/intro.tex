%% -*- mode: latex; mode: reftex; mode: flyspell; TeX-master: "top.tex"; -*-


The  process  of forming  functional  connections  between neurons  is
complex  and   dynamic.   Time-lapse  microscopy   has  revealed  that
differentiating  neurons undergo  a large  range of  dynamic processes
including cell body motility, filopodial dynamics, and repeated cycles
of  neurite growth  and  retraction.  Of  critical  importance is  the
process  by which  axons and  dendrites are  formed in  which a neurite
ceases retracting, extends a   long  distance, and  forms
a connection. Such  dynamic events are  governed by a  complex protein
network that  coordinates   dynamic functions 
within the cytoskeleton, membrane, etc.

Powerful tools such as 
RNA interference  (RNAi)
technology,  fluorescent  protein   labeling,  image  processing,  and
automated high-throughput  microscopy have  opened the door  for large
scale  perturbation studies to help investigate such processes. RNAi  screens have  already led  to novel
insights   into  a  number   of  cellular   processes  such   as  cell
migration~\cite{Bakal07}  and endocytosis~\cite{Collinet10}.  However,
limitations  in  image  processing  have  restricted  most
investigations to static image analysis.

Knowledge  of dynamics is  essential if we are  to understand
complex  processes such  as neuron  morphogenesis.  However, designing
algorithms to quantify dynamic  behaviors is challenging, and
automatic methods  have appeared only  very recently. State-of-the-art
high-throughput techniques have successfully quantified morphodynamics
of  HeLA cancer  cells  in an  effort  to understand  the process  of
mitosis~\cite{Held10,Neumann10,Zhu05}.   However,  the morphology  and
dynamics of  these cell types  are relatively simple compared to neurons,
whose highly deformable  neurites  branch, expand,
retract, and collapse. 


In this paper, we  propose a  fully  automatic method  for  detecting, tracking,  and
segmenting {\em every component of the neuron} (nucleus,  soma, neurites, and filopodia), and quantifying  their dynamic behaviors in ways that were previously not possible. Our approach first  detects nuclei at each time step.  A greedy  tracking algorithm  associates detected nuclei belonging
belonging to the same neuron,  forming a list of detections corresponding  
to that neuron.  Using  the detected nuclei as seed  points, a region-growing
algorithm segments the neuron's soma.  The somata are used to
initialize a joint segmentation of the entire structure of all neurons
appearing in  a image using  a probabilistic method based  on shortest
path computations.   A  graph describing the  morphology of
the  neurites is extracted from  this segmentation.  Each neurite 
tree is tracked by association, and filopodia are detected
by analyzing the topology of the tracked neurites. Finally, 
a set of 156 {\em morphodynamic features} is extracted, quantifying
the behavior of the each neuron in the video.


As   demonstrated  in   Fig.~\ref{fig:video}, 
our  approach produces reliable  segmentations capable of capturing
complex  neuron  dynamics. To validate our approach, we analyzed a   
small-scale  siRNA screen  of 5  genes (3  siRNAs/gene). Our analysis
confirmed steady-state phenotypes obtained previously using 
MetaMorph\texttrademark~\cite{Pertz08}. We were also able to
quantify dynamic behaviors which had been previously observed, but never
measured~\cite{Pertz08}. Our  analysis also uncovered
new dynamic behaviors which are only apparant through dynamic analysis.

%While  our greedy tracking  and probabilistic  segmentation algorithms are  novel,  they are designed to be efficient and thus are relatively simple.  The  main contribution of this  paper is the  system as a whole,  which  is capable  of  high-throughput  processing of  videos, tracking individual  parts of  neurons, and quantifying  their dynamic behaviors in ways that were previously not possible.


%----------------------------------------------------------------------------
\begin{figure}[t]
       \begin{tabular}{@{\hspace{0mm}}c@{}|@{}c@{}}
        \includegraphics[width=45mm] {images/mv1_005.png}  &
        \includegraphics[width=45mm] {images/mv1_008.png} \\ [-.8ex]
        \hline \\ [-2.6ex]
        \includegraphics[width=45mm] {images/mv1_017.png}  &
        \includegraphics[width=45mm] {images/mv1_026.png} \\ [-.8ex]
        \hline \\ [-2.9ex]
       \end{tabular} 
       
      \begin{tabular}{@{\hspace{0mm}}c@{}c@{}c@{}c@{}}
        \includegraphics[width=22.5mm] {images/0_005.png} &
        \includegraphics[width=22.5mm] {images/0_008.png} & 
        \includegraphics[width=22.5mm] {images/0_017.png} & 
        \includegraphics[width=22.5mm] {images/0_026.png} \\ [-1ex]
        \includegraphics[width=22.5mm] {images/2_005.png} &
        \includegraphics[width=22.5mm] {images/2_008.png} & 
        \includegraphics[width=22.5mm] {images/2_017.png} & 
        \includegraphics[width=22.5mm] {images/2_026.png} \\ [-1ex]
        %\includegraphics[width=22.5mm] {images/3_005.png} &
        %\includegraphics[width=22.5mm] {images/3_008.png} & 
        %\includegraphics[width=22.5mm] {images/3_017.png} & 
        %\includegraphics[width=22.5mm] {images/3_026.png} \\ [-1ex]
        {\footnotesize $t = 50$ min} & 
        {\footnotesize $t = 80$ min} & 
        {\footnotesize $t = 170$ min} & 
        {\footnotesize $t = 260$ min} \\ [-1ex]
      \end{tabular}
    \vspace{-2mm}  
    \caption{ {\footnotesize {\it Neuron  Tracking Results.  } The top
        two rows  contain frames from an experiment  where MAP2K7 gene
        functions are inhibited.  For visibility we enhanced the image
        contrast.  Tracked neurons are marked by a unique color and id
        tag.   Nuclei  are denoted  by  filled  ellipsoids, somata  as
        contours,  and neurites  as trees.   Bottom rows  show details
        from  above: 1)  enhanced original  image 2)  tracked neurites
        marked with a different colors. Our  approach   performs  well  even  in  challenging
        situations where neurons appear in close proximity. Note: faintly  stained
        cells are ignored for robustness.}}
    \label{fig:video}
\vspace{-6mm}
\end{figure}
%----------------------------------------------------------------------------


