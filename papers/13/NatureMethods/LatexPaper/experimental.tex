%======================================================================
\subsection{High content imaging of neurite outgrowth dynamics}
%======================================================================
To study neurite outgrowth dynamics, we used the mouse N1E-115 cell neuroblastoma cell line, that can be easily triggered to extend neurites by simple serum starvation and plating on the extracellular matrix protein laminin. To visualize soma and neurite morphology, we used a fusion of green fluorescent protein (GFP) with the F-actin binding peptide lifeact5. This provided high contrast on neurites and soma without excessive accumulation of fluorescence in the thick somata of differentiated N1E-115 cells, allowing for adequate imaging in N1E-115 cells by wide field microscopy (Figure 2a). For unambiguous cell identification, we also simultaneously labeled the nucleus with a mCherry fusion with a nucleus localization sequence (NLS), expressed from the same expression vector than Lifeact-GFP (Supplementary Fig.1a). We observed that transient expression of this construct did not affect neurite outgrowth in N1E-115 cells (Supplementary Fig.1b). To perturb different signaling pathways, we optimized a method to co-transfect the fluorescent marker and siRNAs simultaneously in cells, and observed efficient knockdown in response to knockdown of a panels of mRNAs (Supplementary note 1). To perform high content live cell imaging, we optimized our microscope for fast acquisition of multiple wells of a 24 well plate (Supplementary note 2). This allowed to acquire 240 fields of view in two channels across a 24-well plate with 12 minute time resolution for 20 hours. We performed different sets of experiments with 10 and 20x air objectives. Using 10 x objectives allowed for aquisition of a large field of view with typically 20 objects that could be consistently be observed for long time periods. The 20x objective allowed high resolution imaging of morphological features such as filopodia, but the moving cells often migrated out of the field of view resulting in loss of these objects. Typical 10x and 20x movies are shown (Supplementary movie 1 and 2). In these experiments, we tracked neuronal morphodynamics of differentiated cells that were replated on laminin, allowing to observe almost the whole neurite outgrowth process. We observed that illumination of neurons in the early phase of neurite outgrowth was toxic to the cells, leading to their death. We found that starting the observation process 3 hours post-plating, a time at which cells also had extended 3-4 neurites, allowed the cells to survive until the end of the imaging process.

\textcolor{red}{Describe verbally the neurite outgrowth process here ?}
