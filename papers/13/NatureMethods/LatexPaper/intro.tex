%!TEX root =  main.tex
Neurite outgrowth is essential to build the neuronal processes that connect the adult brain. This morphogenetic process is highly dynamic, and consists of a series of stochastic and repetitive events such as neurite initiation, elongation, branching, growth cone motility and collapse. Each of these different morphogenetic behaviors are likely to be regulated by different signaling networks. Furthermore, all these processes occur on length and time scales of microns and minutes to hours, suggesting an exquisite spatio-temporal regulation of the underlying regulating signals. This is not accessible with approaches in which the effects of molecular perturbations are assessed at the steady-state\cite{Snijder09,Collinet10}, in which only snapshots of intrinsically dynamic behaviors are captured. Rather, understanding the signaling networks that regulate this complex process might benefit from understanding the effect of molecular perturbations on its dynamics.
Advances in automated live cell microscopy now allows for high content image acquisition with high temporal resolution\cite{Held:2010dv}. However, from the analysis side, most computer vision approaches have until now only been used to automatically extract image features from cell populations at the steady-state1,2, rather than from dynamic time-lapse datasets. In one study about mitosis, the temporal context in time-lapse datasets has been taken into account to improve classification of functional cellular states by deconvolving ambiguous, transiently occurring phenotypes with similar morphology3. In another perturbation screen about cell scattering, global dynamic features such as cell migration speed have been measured4. However, no approaches have directly tried to automatically extract information about multiple dynamic cellular features to classify phenotypes.
Here, we present Neurodynamics, an integrated pipeline that allows to study dynamic neurite outgrowth phenotypes in response to molecular perturbations. Neurodynamics consists of three entities: 1. a high content imaging platform for neurite outgrowth dynamics, 2. an automated neuronal shape segmentation algorithm that extracts morphological and morphodynamics features, 3. a feature selection algorithm that allows automatic inference of the neurite outgrowth morphodynamic phenotype. A scheme depicting the global approach is shown in Fig.1. We demonstrate the efficiency of our approach to study a variety of molecular perturbations.