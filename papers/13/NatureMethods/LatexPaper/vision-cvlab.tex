%!TEX root =  main.tex
%======================================================================
\subsection{Automated computer vision analysis of neurite ougrowth dynamics}
%======================================================================
To capture the dynamic neurite outgrowth trajectories of neurons in the native and perturbed state, we developed a computer vision pipeline that allowed to automatically segment and track the soma and neurites in each frame of the timelapse datasets. Our approach first detects nuclei and associated somata at each time step. The nucleus of each neuron is detected as a Maximally Stable Extremal Region (MSER)~\cite{Nister:2008} from the Cherry channel. [As shown in the supplementary note 3, this is more robust than adaptive thresholding.] Using the detected nuclei as seed points, a region-growing algorithm segments the neuron�s soma. Next, the implemented multi-objects tracking algorithm~\cite{BerclazTPAMI2011} searches through the full set of nuclei and somata detections to extract the best K-shortest paths according to a similarity measure between two detections. The proposed similarity measure is derived from the Earth Mover�s Distance~\cite{Pele-iccv2009} between the intensity histograms of the detected somata regions (in the supplementary note 3, we show that this distance provides a good tradeoff between efficiency and precision).  Neurons are detected and tracked at an accuracy of (TODO 95\%) (still waiting for the GT to be completed for number crunching); �Finally, the tracked somata are used to initialize a neurite segmentation and association algorithm based on shortest path computation and Voronoi tessellation. Comparison to manually annotated data demonstrates that �(URGENT TODO: crunch the numbers: still waiting for the GT)�




\comment{
\textcolor{red}{
Fethallah and Kevin can you please write a 10 line long, accessible paragraph that explains the segmentation procedure and the quantification of its efficiency in comparison with the manually annotated ground truth. We can add a supplementary note (Supplementary note 3) that describes the process in depth.
}
Can you please also sketch figure 2 that is dedicated to the segmentation process. This figure will show:
\begin{enumerate}
\item raw green and red images (it is the 1st figure in which we will show this). 
\item schematics of successive steps in the image segmentation.
\item A fully segmented time series of neurons as an example.
\item Some kind of efficiency evaluation in comparison with the manually annotated ground truth.
\end{enumerate}
Please then write another paragraph that explains the different static and dynamic features that are extracted.
\textcolor{red}{
Please sketch figure 3, which explains graphically how some static and dynamic features are extracted. You can represent some of the trajectories of one or two features over time. For this, you can use one of the presentation that Kevin once made. I would try to show a graph of a feature trajectory that is stochastic like the protrusion/retraction cycles of neurite outgrowth. This shows that we look at highly dynamic events, and thus illustrates the benefit of our approach. We can add a number of supplementary movies with different morphological features that are analyzed (soma, each new neurites). Please prepare also supplementary tables about the definition of features (supp tables 1 and 2). Then we also need a supplementary note that explain the format in which the morphodynamic history of each cell is explained.}
}