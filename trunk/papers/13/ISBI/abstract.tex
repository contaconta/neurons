%% -*- mode: latex; mode: reftex; mode: flyspell; TeX-master: "top.tex"; -*-

We present a fully automatic method to track and
quantify the morphodynamics of differentiating
neurons in fluorescence time-lapse datasets.
Previous high-throughput studies have been limited to
static analysis or simple behavior. Our approach
opens the door to rich dynamic analysis of complex
cellular behavior in high-throughput time-lapse
data. It is capable of robustly detecting,
tracking, and segmenting all the components of the
neuron including the nucleus, soma, neurites, and
filopodia. It was designed to be efficient enough
to handle the massive amount of data from a
high-throughput screen. Each image is processed in
approximately two seconds on a notebook
computer. To validate the approach, we analyzed
over 500 neuronal differentiation videos from a
small-scale RNAi screen. Our fully automated
analysis of over 7,000 neurons quantifies and
confirms with strong statistical significance
static and dynamic behaviors that had been
previously observed by biologists, but never
measured.



\begin{keywords}
Molecular and cellular screening; Image sequence processing; Fluorescence microscopy
\end{keywords}



%Finally, we are first to
%analyze temporally the  behavior of neurons, leading to  findings not measurable
%through  state-of-the-art static  analysis. All  our findings  are statistically
%significant, with a p-value $\ll 0.05$.

%% We  present  a  fully  automatic  method to  track  and  quantify  the
%% morphodynamics of  differentiating neurons in  fluorescence time-lapse
%% datasets.     Our  approach is capable of robustly detecting,
%% tracking, and segmenting all the  components of the neuron including 
%% the nucleus, soma, neurites, and filopodia.  It is designed to be extremely efficient, capable of processing
%% each image in approximately two seconds on a conventional notebook
%% computer.  To validate our approach, we  analyzed  neuronal
%% differentiation videos in  which a set of genes  was perturbed using
%% RNA interference. Our analysis quantifes and confirms morphodynamic behaviors 
%% which had been previously observed by biologists but never measured.
%% Finally, we present new observations on the behavior of neurons made
%% possible by our analysis which could not be discovered through static analyis.



%  whose
%measurements were not feasible on a large scale.  Finally, using our
%quantitative analysis we make

%our dynamic
%quantitative analysis allows us to  make new observations that are not
%visible to a human observer.


%% we  use it to  perform an siRNA  screen and
%% successfully reproduce the findings of~\cite{}, which was performed at
%% steady-state under less stressful conditions [OLVIER, HELP US SAY THIS
%%   MORE  ELOQUENTLY].   Finally, we  apply  our  approach  to make  new
%% observations   based  on  dynamic   quantitative  analysis   that  was
%% previously impossible.

%% While  previous efforts
%% have tracked the soma and nucleus,  our approach is the first to fully automatically track
%% and reconstruct the neurites as  they expand, branch and collapse, and
%% to dynamically detect fine filopodia structures.

%We present a method to analyze and extract statistics from large-scale
%video sequences  of in-vitro neurons.   Our system is able  to detect,
%track, segment  and reconstruct each individual neuron  present on the
%videos.   We show how  such system  can be  used to  find correlations
%between neuron behaviour and the proteins used to mutate them.

%high throughput automation

%confirms pre-existing findings

%yields new understandings about the dynamic morphologies

%efficient, tracks, segments, and tracks neurites

