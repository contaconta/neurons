We have described a  fully  automatic  method to  track  and  quantify  the
morphodynamics of  differentiating neurons in  fluorescence time-lapse
datasets.     Our  approach is capable of robustly detecting,
tracking, and segmenting the entire neuron including 
the nucleus, soma, neurites, and filopodia.  Previous efforts to analyze
high-throughput screens have been limited to static images or simple cell
behavior, whereas our approach provides researchers with a rich dynamic 
analysis of complex cellular behavior in high-throughput time-lapse data.
From the rich set of 156 features we extract in our experiments we are able to
 to 1) corroborate previous findings by biologists, 2) quantify previously observed neuronal behavior and 3) infer new unobserved behaviour, all with strong statistical significance.
In the future, we plan to expand this
work to a larger scale screen and develop statistical techniques to 
infer relationships between genes and morphodynamic phenotypes.




%Our approach is 
%extremely efficient, and is capable of quantifying morphodynamics
%of neurons through 156 meaningful features. Using our approach, we were 
%able to reproduce previous findings from a static analysis, 
%quantify behaviors that had been previously observed but never measured,
%and uncover dynamic phenotypes. In the future, we plan to expand this
%work to a larger scale screen and develop statistical techniques to 
%infer relationships between genes and morphodynamic phenotypes.



%It is designed to be extremely efficient, capable of processing
%each image in approximately two seconds on a conventional notebook
%computer.  To validate our approach, we  analyzed  neuronal
%differentiation videos in  which a set of genes  was perturbed using
%RNA interference. Our analysis quantifes and confirms morphodynamic behaviors 
%which had been previously observed by biologists but never measured.
%Finally, we present new observations on the behavior of neurons made
%possible by our analysis which could not be discovered through static analyis.



%We have described a  set of algorithms
%which, as a  system, are capable of robustly  tracking and segmenting entire
%neurons  including  the nucleus,  soma,  neurites  and filopodia.  Our
%approach is efficient, and  can analyze high-throughput datasets using
%meaningful dynamic  features extracted from  our segmentations.
%We  validated  our  approach  by  reproducing
%previous findings,  confirming anecdotal findings,  and uncovering new
%dynamic phenotypes. It is beyond the scope of this work to comment on
%the biological significance of these findings, if there is any. We
%leave that for future work.
