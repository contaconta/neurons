%% -*- mode: latex; mode: reftex; mode: flyspell; TeX-master: "top.tex"; -*-
\vspace{-3mm} Our neuron tracking, segmentation
and delineation method produces sets of graphs
linking detections, contours, and trees to define
each neuron over time.  This data structure is not
immediately useful for quantifying dynamic
behaviors.  To facilitate the analysis, we extract
a set of {\em 156 informative features} from our
data structure to quantify morphodynamics, which
are too numerous to list here.  A few examples for
the nucleus and soma include: area, perimeter,
Lifeact-GFP intensity, NLS-mCherry intensity,
speed, acceleration, total distance traveled, time
spent expanding/contracting, frequency of
expansion.  For neurites: number of branches,
distance from tip to soma, filopodia length,
number of filopodia, major axis, minor axis and
eccentricity of an ellipse fitted to the neurite,
total length, time spent expanding/contracting,
frequency of expansion. We also compute
change-over-time for each of the features
mentioned above (denoted by $\Delta$).

